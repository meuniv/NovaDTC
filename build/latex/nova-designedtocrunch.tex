%% Generated by Sphinx.
\def\sphinxdocclass{report}
\documentclass[letterpaper,10pt,english]{sphinxmanual}
\ifdefined\pdfpxdimen
   \let\sphinxpxdimen\pdfpxdimen\else\newdimen\sphinxpxdimen
\fi \sphinxpxdimen=.75bp\relax

\PassOptionsToPackage{warn}{textcomp}
\usepackage[utf8]{inputenc}
\ifdefined\DeclareUnicodeCharacter
% support both utf8 and utf8x syntaxes
  \ifdefined\DeclareUnicodeCharacterAsOptional
    \def\sphinxDUC#1{\DeclareUnicodeCharacter{"#1}}
  \else
    \let\sphinxDUC\DeclareUnicodeCharacter
  \fi
  \sphinxDUC{00A0}{\nobreakspace}
  \sphinxDUC{2500}{\sphinxunichar{2500}}
  \sphinxDUC{2502}{\sphinxunichar{2502}}
  \sphinxDUC{2514}{\sphinxunichar{2514}}
  \sphinxDUC{251C}{\sphinxunichar{251C}}
  \sphinxDUC{2572}{\textbackslash}
\fi
\usepackage{cmap}
\usepackage[T1]{fontenc}
\usepackage{amsmath,amssymb,amstext}
\usepackage{babel}



\usepackage{times}
\expandafter\ifx\csname T@LGR\endcsname\relax
\else
% LGR was declared as font encoding
  \substitutefont{LGR}{\rmdefault}{cmr}
  \substitutefont{LGR}{\sfdefault}{cmss}
  \substitutefont{LGR}{\ttdefault}{cmtt}
\fi
\expandafter\ifx\csname T@X2\endcsname\relax
  \expandafter\ifx\csname T@T2A\endcsname\relax
  \else
  % T2A was declared as font encoding
    \substitutefont{T2A}{\rmdefault}{cmr}
    \substitutefont{T2A}{\sfdefault}{cmss}
    \substitutefont{T2A}{\ttdefault}{cmtt}
  \fi
\else
% X2 was declared as font encoding
  \substitutefont{X2}{\rmdefault}{cmr}
  \substitutefont{X2}{\sfdefault}{cmss}
  \substitutefont{X2}{\ttdefault}{cmtt}
\fi


\usepackage[Bjarne]{fncychap}
\usepackage{sphinx}

\fvset{fontsize=\small}
\usepackage{geometry}


% Include hyperref last.
\usepackage{hyperref}
% Fix anchor placement for figures with captions.
\usepackage{hypcap}% it must be loaded after hyperref.
% Set up styles of URL: it should be placed after hyperref.
\urlstyle{same}
\addto\captionsenglish{\renewcommand{\contentsname}{Contents:}}

\usepackage{sphinxmessages}
\setcounter{tocdepth}{1}



\title{Nova \sphinxhyphen{} Designed to Crunch}
\date{Nov 23, 2020}
\release{}
\author{Vincent Meunier}
\newcommand{\sphinxlogo}{\vbox{}}
\renewcommand{\releasename}{}
\makeindex
\begin{document}

\pagestyle{empty}
\sphinxmaketitle
\pagestyle{plain}
\sphinxtableofcontents
\pagestyle{normal}
\phantomsection\label{\detokenize{index::doc}}



\chapter{Welcome to Designed to Crunch}
\label{\detokenize{introduction:welcome-to-designed-to-crunch}}\label{\detokenize{introduction:introduction}}\label{\detokenize{introduction::doc}}
This Nova award is designed to help you explore how math affects your life each day.

The site is organized per requirement. To succesfully complete the requirements, the scout will get in touch with an approved Nova Counselor and give proofs of completion.

Before you start anything, make sure you identify a counselor and read carefully the section below on how to complete and document your progress!


\section{Documenting your progress}
\label{\detokenize{introduction:documenting-your-progress}}
A template worksheet can be found \sphinxhref{bdbdbd}{here}. This is a \sphinxstyleemphasis{Google document}. You will not be able to modify it until you make your own copy as I will now describe for you.


\section{If you have any question}
\label{\detokenize{introduction:if-you-have-any-question}}
Contact your counselor or your scoutmaster! If you have questions about the program, contact Dr. Meunier  by \sphinxhref{mailto:vinmeunier@gmail.com}{email}.

\noindent{\hspace*{\fill}\sphinxincludegraphics[scale=0.5]{{logo4}.png}\hspace*{\fill}}

\begin{sphinxadmonition}{note}{Note:}
Most of the material used here was obtained from a number of external scouting sources, including \sphinxhref{https://www.scouting.org/wp-content/uploads/2018/11/Designed-to-Crunch-Nova-2018Nov26.pdf}{scouting.org}
\end{sphinxadmonition}


\chapter{Requirement \#1}
\label{\detokenize{requirement1:requirement-1}}\label{\detokenize{requirement1::doc}}
Choose A or B or C or D and complete ALL the requirements.
\begin{enumerate}
\sphinxsetlistlabels{\Alph}{enumi}{enumii}{}{.}%
\item {} 
Watch about three hours total of math\sphinxhyphen{}related shows or documentaries that involve scientific models and modeling, physics, sports equipment design, bridge building, or cryptography. Then do the following:
\begin{enumerate}
\sphinxsetlistlabels{\arabic}{enumii}{enumiii}{(}{)}%
\item {} 
Make a list of at least five questions or ideas from the show(s) you watched.

\item {} 
Discuss two of the questions or ideas with your counselor.

\end{enumerate}

\begin{sphinxadmonition}{tip}{Tip:}
Some examples include—but are not limited to—shows found on PBS (“NOVA”), Discovery Channel, Science Channel, National Geographic Channel, TED Talks (online videos), and the History Channel. You may choose to watch a live performance or movie at a planetarium or science museum instead of watching a media production. You may watch online productions with your counselor’s approval and under your parent’s supervision.
\end{sphinxadmonition}

\item {} 
Research (about three hours total) several websites (with your parent’s or guardian’s permission) that discuss and explain cryptography or the discoveries of people who worked extensively with cryptography. Then do the following:
\begin{enumerate}
\sphinxsetlistlabels{\arabic}{enumii}{enumiii}{(}{)}%
\item {} 
List and record the URLs of the websites you visited and major topics covered on the websites you visited. (You may use the copy and paste function— eliminate the words—if you include your sources.)

\item {} 
Discuss wit hyour counselor how cryptography is used in the military and in everyday life and how a cryptographer uses mathematics.

\end{enumerate}

\begin{sphinxadmonition}{tip}{Tip:}
“The Mathematics of Cryptology”: University of Massachusetts Website: \sphinxurl{http://www.math.umass.edu/~gunnells/talks/crypt.pdf}
\end{sphinxadmonition}

\item {} 
Read at least three articles (about three hours total) about physics, math, modeling, or cryptography. You may wish to read about how technology and engineering are changing sports equipment, how and why triangles are used in construction, bridge building, engineering, climate and/or weather models, how banks keep information secure, or about the stock market. Then do the following:
\begin{enumerate}
\sphinxsetlistlabels{\arabic}{enumii}{enumiii}{(}{)}%
\item {} 
Make a list of at least two questions or ideas from each article

\item {} 
Discuss two of the questions or ideas with your counselor.

\end{enumerate}

\begin{sphinxadmonition}{tip}{Tip:}
Examples of magazines include—but are not limited to—Odyssey, Popular Mechanics, Popular Science, Science Illustrated, Discover, Air \& Space, Popular Astronomy, Astronomy, Science News, Sky \& Telescope, Natural History, Robot, Servo, Nuts and Volts, and Scientific American.
\end{sphinxadmonition}

\item {} 
Do a combination of reading, watching, and researching (about three hours total). Then do the following:
\begin{enumerate}
\sphinxsetlistlabels{\arabic}{enumii}{enumiii}{(}{)}%
\item {} 
Make a list of at least two questions or ideas from each article, website, or show.

\item {} 
Discuss two of the questions or ideas with your counselor.

\end{enumerate}

\end{enumerate}

\begin{sphinxadmonition}{attention}{Attention:}
Once you have completed this requirement, make sure you document it in your worksheet!
\end{sphinxadmonition}


\chapter{Requirement \#2}
\label{\detokenize{requirement2:requirement-2}}\label{\detokenize{requirement2::doc}}
Complete ONE merit badge from the following list. Choose one that you have not already used toward another Nova award.
After completion, discuss with your counselor how the merit badge you earned uses mathematics.
\begin{itemize}
\item {} 
American Business

\item {} 
Chess

\item {} 
Computers

\item {} 
Digital Technology

\item {} 
Drafting

\item {} 
Entrepreneurship

\item {} 
Orienteering

\item {} 
Personal Management

\item {} 
Radio

\item {} 
Signs, Signals, and Codes

\item {} 
Surveying

\item {} 
Weather

\end{itemize}

\begin{sphinxadmonition}{attention}{Attention:}
Once you have completed this requirement, make sure you document it in your worksheet!
\end{sphinxadmonition}


\chapter{Requirement \#3}
\label{\detokenize{requirement3:requirement-3}}\label{\detokenize{requirement3::doc}}
Choose TWO from A or B or C or D or E and complete ALL the requirements for the two you choose. (Write down your data and calculations to support your explanation to your counselor. You may use a spreadsheet. Do not use someone else’s data or calculations.)
\begin{enumerate}
\sphinxsetlistlabels{\Alph}{enumi}{enumii}{}{.}%
\item {} 
Calculate your horsepower when you run up a flight of stairs.
\begin{enumerate}
\sphinxsetlistlabels{\arabic}{enumii}{enumiii}{(}{)}%
\item {} 
How does your horse power compare to the power of a horse?

\item {} 
How does your horse power compare to the horse power of your favorite car?

\end{enumerate}

\end{enumerate}

\begin{sphinxadmonition}{tip}{Tip:}
Helpful Links: “How to Calculate Your Horsepower”: wikiHow  Website: \sphinxurl{http://www.wikihow.com/Calculate-Your-HorsepowerHaplosciences.net} Website: \sphinxurl{http://onlinephys.com/labpower1.html}
\end{sphinxadmonition}

\begin{sphinxadmonition}{note}{Note:}
Share your calculations with your counselor, and discuss what you learned about horsepower.
\end{sphinxadmonition}
\begin{enumerate}
\sphinxsetlistlabels{\Alph}{enumi}{enumii}{}{.}%
\setcounter{enumi}{1}
\item {} 
Attend at least two track, cross country, or swim meets.
\begin{enumerate}
\sphinxsetlistlabels{\arabic}{enumii}{enumiii}{(}{)}%
\item {} 
For each meet,time at least three racers.(Time the same racers at each meet.)

\item {} 
Calculate the average speed of the racers you timed.(Make sure you record your data and calculations.)

\item {} 
Compare the average speeds of your racers to each other,to the official time, and to their times at the two meets you attended.

\end{enumerate}

\end{enumerate}

\begin{sphinxadmonition}{note}{Note:}
Share your calculations with your counselor, and discuss your conclusions about the racers’ strengths and weaknesses.
\end{sphinxadmonition}
\begin{enumerate}
\sphinxsetlistlabels{\Alph}{enumi}{enumii}{}{.}%
\setcounter{enumi}{2}
\item {} 
Attend a soccer, baseball, softball, or basketball game. Then choose two players. Keep track of their efforts during the game. (Make sure you record your data and calculations.) Calculate their statistics using the following as examples:
\begin{enumerate}
\sphinxsetlistlabels{\arabic}{enumii}{enumiii}{(}{)}%
\item {} 
Soccer—Goals, assists, cornerkicks, keepersaves, fouls, offsides

\item {} 
Baseball or softball—Batting average,runs batted in,fielding statistics, pitching statistics

\item {} 
Basketball—Points, baskets attempted, rebounds, steals, turnovers, and blocked shots

\end{enumerate}

\end{enumerate}

\begin{sphinxadmonition}{note}{Note:}
Share your calculations with your counselor, and discuss your conclusions about the players’ strengths and weaknesses.
\end{sphinxadmonition}
\begin{enumerate}
\sphinxsetlistlabels{\Alph}{enumi}{enumii}{}{.}%
\setcounter{enumi}{3}
\item {} 
Attend a football game or watch one on TV. (This is a fun activity to do with a parent or friend.) Keep track of the efforts of your favorite team during the game. (Make sure you record your data and calculations.) Then calculate your team’s statistics using the following as examples:
\begin{enumerate}
\sphinxsetlistlabels{\arabic}{enumii}{enumiii}{(}{)}%
\item {} 
Kicks/punts
\begin{enumerate}
\sphinxsetlistlabels{\alph}{enumiii}{enumiv}{(}{)}%
\item {} 
Kickoff—Kick return yards

\item {} 
Punt—Number, yards

\item {} 
Field goals—Attempted, percent completed, yards

\item {} 
Extrapoints—Attempted, percent completed

\end{enumerate}

\item {} 
Offense
\begin{enumerate}
\sphinxsetlistlabels{\alph}{enumiii}{enumiv}{(}{)}%
\item {} 
Number of first downs

\item {} 
Forward passes—Attempted, percent completed, total length of passes, longest pass, number and length of passes caught by each receiver, yardage gained by each receiver after catching a pass

\item {} 
Running plays—Number, yards gained or lost for each run, longest run from scrimmage line, total yards gained or lost, and number of touchdowns

\end{enumerate}

\item {} 
Defense—Number of quarterback sacks, interceptions, turnovers, and safeties

\end{enumerate}

\end{enumerate}

\begin{sphinxadmonition}{note}{Note:}
Share your calculations with your counselor, and discuss your conclusions about your team’s strengths and weaknesses.
\end{sphinxadmonition}
\begin{enumerate}
\sphinxsetlistlabels{\Alph}{enumi}{enumii}{}{.}%
\setcounter{enumi}{4}
\item {} 
How starry are your nights? Participate in a star count to find out. This may be done alone but is more fun with a group.
\begin{enumerate}
\sphinxsetlistlabels{\arabic}{enumii}{enumiii}{(}{)}%
\item {} 
Visit the website of the Astronomical Society of the Pacific at www.astrosociety.org/education/hands\sphinxhyphen{}on\sphinxhyphen{}astronomy\sphinxhyphen{}activities for instructions on performing a star count.

\item {} 
Do a star count on five clear nights at the same time each night.

\end{enumerate}

\end{enumerate}

\begin{sphinxadmonition}{note}{Note:}
Afterward, share and discuss your results with your counselor.
\end{sphinxadmonition}

\begin{sphinxadmonition}{attention}{Attention:}
Once you have completed this requirement, make sure you document it in your worksheet!
\end{sphinxadmonition}


\chapter{Requirement \#4}
\label{\detokenize{requirement4:requirement-4}}\label{\detokenize{requirement4::doc}}\begin{enumerate}
\sphinxsetlistlabels{\arabic}{enumi}{enumii}{}{.}%
\setcounter{enumi}{3}
\item {} 
Do ALL of the following:
A. Investigate your calculator and explore the different functions.
B. Discuss the functions, abilities, and limitations of your calculator with your counselor. Talk about how these affect what you can and cannot do with a calculator. (See your counselor for some ideas to consider.)

\end{enumerate}

\begin{sphinxadmonition}{attention}{Attention:}
Once you have completed this requirement, make sure you document it in your worksheet!
\end{sphinxadmonition}


\chapter{Requirement \#5}
\label{\detokenize{requirement5:requirement-5}}\label{\detokenize{requirement5::doc}}\begin{enumerate}
\sphinxsetlistlabels{\arabic}{enumi}{enumii}{}{.}%
\setcounter{enumi}{4}
\item {} 
Discuss with your counselor how math affects your everyday life.

\end{enumerate}

\begin{sphinxadmonition}{attention}{Attention:}
Once you have completed this requirement, make sure you document it in your worksheet!
\end{sphinxadmonition}

\noindent{\hspace*{\fill}\sphinxincludegraphics[scale=0.5]{{logo4}.png}\hspace*{\fill}}



\renewcommand{\indexname}{Index}
\printindex
\end{document}